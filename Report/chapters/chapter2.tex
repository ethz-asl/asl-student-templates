\chapter{Richtiges technisches~Schreiben}
\label{sec:technischesSchreiben}

DIN 5008, Schreib- und Gestaltungsregeln für die Textverarbeitung, DIN 1333, 
Zahlenangaben, und ISO 80000 bzw. IEC 80000 in dreizehn Teilen regeln die 
Schreibweise von Masszahlen und Masseinheiten, Währungen, Abkürzungen, 
Zwischenräumen und Satzzeichen -- im Deutschen und im Englischen. 
In der Typografie nennt man dieses Feld „Mikro\-typografie“. \\

Nützliche Online-Referenzen sind:
\begin{description}
  \item[--] \href{https://din-5008-richtlinien.de}{Online Referenz zu DIN 5008}
  \item[--] \href{https://de.wikipedia.org/wiki/Schreibweise_von_Zahlen}{Wikipedia: Schreibweise von Zahlen}
  \item[--] \href{https://de.wikibooks.org/wiki/LaTeX-Kompendium:_Sonderzeichen}{LaTeX-Kompendium: Sonderzeichen}
\end{description}

\section{Unterschied: Trenn-, Binde-, Gedankenstrich}

\subsection{Trennstrich}

Trennstriche dienen im Deutschen und Englischen der Wortrennung (Textumbruch).
Sie werden von jedem Text\-verarbeitungsprogramm automatisch erzeugt, sobald die 
Silbentrennung aktiv ist.

\subsection{Bindestrich} 

Bindestriche werden verwendet für Worverbindungen speziell mit
Fremdwortkomponente. Beispiele:
\begin{description}
  \item[--] Mehrzweck-Maschine
  \item[--] Shopping-Center
  \item[--] (engl.) state-of-the-art
  \item[--] (engl.) seven-year-old
\end{description}

\subsection{Gedankenstrich, Halbgeviertstrich} 

Ein Gedankenstrich unterbricht die Syntax: „Plötzlich -- 
ein gellender Aufschrei!“, „Das Bild -- sein bekanntestes -- 
wurde verkauft.“ \\

Der Gedankenstrich ist länger als der  Trenn- und Bindestrich. 
Typografisch heisst der längere dieser Striche „Halbgeviertstrich“. 
Der Halbgeviertstrich wird über die Tastatur als Sonderzeichen eingegeben 
(Mac: option + -, Win: Strg + Alt + -). In LaTeX wird der 
Halbgeviertstrich durch zwei aufeinander folgende Bindestriche erzeugt

\subsection{Geviertstrich} 

Der Geviertstrich wird im Englischen als Gedankenstrich verwendet. 
Beispiel: „My son---the teacher---would like to meet you.“ 
In LaTeX wird der Geviertstrich durch drei aufeinander folgende 
Bindestriche erzeugt.

\subsection{Bis oder gegen: ein kleiner aber wichtiger Unterschied}

Der Halbgeviertstrich wird auch als Minuszeichen und als Abkürzung für 
„bis“ und „gegen“ verwendet. Der Unterschied liegt bei den Leerzeichen: 
\hfill \break
16–18 Uhr (bis, ohne Leerzeichen), aber: FC Zürich – GC 
(gegen, mit Leerzeichen) \\

Zur Vertiefung: 
\begin{description}
  \item[--] \href{https://typefacts.com/artikel/binde-und-gedankenstrich}{Typefacts.com} 
  \item[--] \href{https://de.wikihow.com/Gedankenstrich-und-Bindestrich-in-der-englischen-Sprache}{Wikihow: Gedankenstrich und Bindestrich in der englischen Sprache}
\end{description}

\subsection{Schrägstrich („Slash“)}

Mitarbeiter/-innen (im Englischen und Deutschen ohne Leerzeichen)

\section{Schreibweise von Zahlen}

Ein- und zweisilbige Zahlen im Lauftext werden ausgeschrieben -- 
aber nie im Zusammenhang mit einem Mass. Beispiel: „Er ist dreizehn 
Jährig und immer noch 1\,m gross.“

\subsection{Gliederung von Zahlen}

Grundsätzlich gilt:
\begin{description}
  \item[--] CH/D: Dreierblöcke (aber nicht bei vier Stellen): 43\,300\,000, aber 4000
  \item[--] UK/USA: 43,300,000 und 4,000
\end{description}

\subsection{Trennung von Vor- und Nachkommastellen}

Je nach Sprache und Land werden Vor- und Nachkommastellen mit Punkt 
oder Komma getrennt: 
\begin{description}
  \item[--] CH: Trennung mit Punkt oder Komma: 4.3 oder 4,3
  \item[--] D: Trennung mit Komma: 4,3
  \item[--] UK/USA: Trennung mit Punkt (Komma ist für Gliederung reserviert)
\end{description}

\subsection{Gliederung von Nachkommastellen}
–43\,300.060\,395, –43\,300.0603, –43\,300.060\,39

\subsection{Negative Zahlen}

Das Minuszeichen ist ein Geviert- und kein Trennungsstrich 
(siehe auch weiter unten).
Zwischen Minuszeichen und Zahl steht kein Leerzeichen:
\begin{description}
 \item[--] richtig: –4
 \item[--] falsch: -4
 \item[--] falsch: – 4
\end{description}

\subsection{Zwei typografische Anforderungen an Leerzeichen}

Gliedernde Leerzeichen zwischen Zahlenblöcken, Mass und Einheit und in
Abkürzungen müssen umbruchgeschützt sein und sind schöner anzusehen, wenn
sie schmal sind. \\

Zum Vergleich:
\begin{description}
  \item[--] 123 456 N/cm\textsuperscript{2} (mit Leerzeichen)
  \item[--] 123\,456 N/cm\textsuperscript{2} (mit halbem Leerschlag: 
  schöner und umbruchgeschützt)
\end{description}

Das umbruchgeschützte schmale Leerzeichen wird in LaTeX wie folgt erzeugt: 
\begin{verbatim}13,7\,m\end{verbatim}

Ein umbruchgeschütztes Leerzeichen, das man auch zur Steuerung der Silbentrennung
einsetzen kann wird mit einem Tilde eingegeben:
\begin{verbatim}Prof.~Dr.~R.~Siegwart\end{verbatim}

\section{Masszahlen}

Zwischen Zahl und Mass steht das umbruchgeschützte halbe Leerzeichen. Ausnahmen 
sind Winkelmasse: 37° (hingegen regelkonsistent: 37 °C) und Prozentangaben im 
Englischen: 0,37\% (im Deutschen allerdings mit Leerschlag: 0,37\,\%). \\

Für LaTeX gibt es ein Paket siunitx, mit dem sich SI-Einheiten wie folgt 
darstellen lassen:
\begin{verbatim}
\si{kg.m/s^2} 
\end{verbatim}

Das ergibt: kg\,m/s2 \\

Download und Installationshinweise:
\begin{description}
\item[--] \href{https://ctan.org/pkg/siunitx?lang=de}{siunitx Download}
\item[--] \href{https://mirror.kumi.systems/ctan/macros/latex/contrib/siunitx/siunitx.pdf}{Dokumentation: siunitx}
\end{description}

\section{Währungen}
\subsection{Abkürzungen und Währungssymbole}

Der Franken kann wie folgt abgekürzt werden: Fr., sFr. (beide mit Punkt), 
sfr oder CHF (beide ohne Punkt). \\

Für Euro und Dollar gibt es Sonderzeichen: €555 oder \$555. \\

In LaTeX geben Sie das Euro-Zeichen mit der Tastatur ein und das Dollar-Zeichen 
mit Backslash und \$.

\subsection{Stellung}
Die Währungseinheit kann vor oder nach der Zahl stehen. Der Duden-Newsletter sagt: 
„In Fließtexten ist zwar die letztgenannte Variante zu empfehlen (250 EUR), 
da diese Schreibweise dem Lesefluss entspricht, ansonsten bleibt die Entscheidung 
Ihnen überlassen.“

\subsection{Euro,Cent – Franken.Rappen}

Euro und Cent werden mit Komma getrennt, Franken und Rappen mit Punkt: \hfill \break
19,90\,€ bzw. 19.90\,sFr. \\

Richtig (deutscher Text) sind zum Beispiel:
\begin{description}
  \item[--] 400 Franken
  \item[--] 400\,Fr. 
  \item[--] 400\,sFr.
  \item[--] 400\,sfr 
  \item[--] 400\,CHF
  \item[--] Fr.\,400
  \item[--] sFR.\,400 
  \item[--] sfr\,400
  \item[--] CHF\,400 
  \item[--] €\,59,–
  \item[--] 59\,€
\end{description}


Falsch (deutscher Text) sind zum Beispiel:
\begin{description}
  \item[--] Vierhundert Fr. (wird nicht ausgeschrieben -- wie bei Masszahlen)
  \item[--] 307.30 Euro (Euro,Cent wäre richtig)
  \item[--] sFr\,307.30 (Punkt fehlt nach sFr)
\end{description}

Richtig im Englischen sind:
\begin{description}
  \item[--] US\$50
  \item[--] US\$50.35
  \item[--] €4.50
\end{description}

\section{Anführungszeichen}

Richtige Anführungszeichen sind:
\begin{description}
  \item[--] im Deutschen: Herbert sagt: „Hallo!“ (99-66, unten-oben)
  \item[--] im Englischen: Herbert says: ``Hello!'' (66-99, oben-oben)
\end{description}

In LaTeX geben Sie die englischen Zeichen wie folgt ein:
\begin{verbatim}
``Text'' und `Text'
\end{verbatim}

Die deutschen Zeichen hingegen geben Sie via Tastatur ein oder wie folgt 
(Voraussetzung ist, das Babel eingebunden ist):
\begin{verbatim}
\usepackage[ngerman]{babel}

"`Text"' oder \glqq Text\grqq
\end{verbatim}

Die Befehle für die äusseren und inneren Anführungszeichen im Deutschen lauten:
\begin{verbatim}
Er sagte: \glqq Das Buch heisst: \glq Einführung in das 
technische Schreiben\grq.\grqq (äussere, doppelte und 
innere, einfache Anführungszeichen.
\end{verbatim}

\section{Apostroph}

Der Apostroph ist nicht das gerade Fusszeichen – LaTeX macht aber aus der 
Tastatur\-eingabe für „foot“ automatisch einen richtigen Apostroph 
(Kringel in der Form einer neun, oben). \hfill \break
Beispiel: „Das war 'ne Katastrophe!“